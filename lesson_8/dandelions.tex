\begin{problem}{Одуванчики}{dandelions.in}{dandelions.out}{2 секунды}
{256 мегабайт}

На плоскости даны $N$~одуванчиков. Вам требуется построить выпуклую оболочку данного множества одуванчиков и~вывести длину её периметра.
Предполагается решение за $O(N \cdot \log{N})$. 

Алгоритм Грэхема --- алгоритм построения выпуклой оболочки в двумерном пространстве. 
В этом алгоритме задача решается с помощью стека, сформированного из точек-кандидатов. 
Все точки входного множества заносятся в стек, а потом точки, не являющиеся вершинами выпуклой оболочки, 
со временем удаляются из него. 
По завершении работы алгоритма в стеке остаются только вершины оболочки в порядке их обхода против часовой стрелки.

Входные данные --- множество точек $Q$, где $|Q|\geqslant 3$. 
В ней вызывается функция Top(S), которая возвращает точку, 
находящуюся на вершине стека $S$, не изменяя при этом его содержимое. 
Кроме того, используется также функция NextToTop(S), 
которая возвращает точку, расположенную в стеке $S$, 
на одну позицию ниже от верхней точки; стек $S$ при этом не изменяется.


\begin{enumerate}
 \item Пусть $p_0$ — точка из множества Q с минимальной координатой y или самая левая из таких точек при наличии совпадений
 \item Пусть $p_1, p_2,\ldots,p_m$ — остальные точки множества Q, отсортированные в порядке возрастания полярного угла,
       измеряемого против часовой стрелки относительно точки $p_0$ 
      (если полярные углы нескольких точек совпадают, то по расстоянию до точки $p_0$)
 \item Push($p_0$,S)
 \item Push($p_1$,S)
 \item '''for''' i = 2 '''to''' m '''do'''
 \item     '''while''' угол, образованный точками NextToTop(S), Top(S) и $p_i$, образуют не левый поворот
             (при движении по ломаной, образованной этими точками, мы движемся прямо или вправо)
 \item       '''do''' Pop(S)
 \item    Push($p_i$,S)
 \item '''return''' S
\end{enumerate}

Для определения, образуют ли три точки $a$, $b$ и $c$ 
левый поворот, можно использовать векторное произведение, а именно условие 
левого поворота будет выглядеть следующим образом: 
$u_x v_y - u_y v_x > 0$, где $ u = \left\{ b_x - a_x, \; b_y - a_y \right\},
v = \left\{ c_x - a_x, \; c_y - a_y \right\}$

\InputFile
Первая строка содержит количество одуванчиков $N$ ($1 \leqslant N \leqslant 20\,000$).
Каждая из~последующих $N$~строк содержит два целых числа~--- координаты 
$x_i$ и~$y_i$. Координаты по~модулю не~превосходят $10\,000$.

\OutputFile
Выведите в~выходной файл длину периметра выпуклой оболочки с~максимально возможной точностью.
Если в~выпуклой оболочке 2~одуванчика, то требуется вывести удвоенную длину отрезка.


\Example

\begin{example}
\exmp{5
0 0
1 0
0 1
-1 0
0 -1
}{5.65685}%
\end{example}

\end{problem}
